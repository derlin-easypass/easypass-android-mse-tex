\section{Difficultés rencontrées}

Dans ce projet nous avons décidé de partir sur un nouveau langage et de tester de nouvelles choses jusque-là inconnues. Cela a demandé un certain temps, mais ne nous a pas posé de problèmes majeurs.

Les parties les plus difficiles, outre l'utilisation des clés privées et des empreintes, fut sûrement l'utilisation des styles. Nous avons également passé beaucoup de temps à peaufiner l'aspect visuel (comparé au temps passé à implémenter la logique applicative).

Parmis les difficultés rencontrées, nous voudrions citer la gestion du \emph{soft keyboard} (comment savoir/réagir quand il apparait/disparait ?) et la gestion du cycle de vie de l'application (comment savoir combien de temps une application est en \emph{background}?, comment redémarrer complètement une application?). Ces problèmes sont assez généraux et touchent de nombreuses applications dans différents contextes. Cependant, elles n'ont aucune solution directe. Il existe des \emph{hacks} sur internent, mais après en avoir essayé beaucoup, aucun ne marchait réellement sans casser autre chose... Bref, nous avons finalement dû contourner les problèmes autrement. 

\section{Perspectives futures}

Nous pensons avoir développé une application intéressantes. Nous envisageons de la rendre disponible sur le Play Store dans les prochains mois en mode bêta. 

L'étape suivante serait de la rendre public. Cela demande cependant quelques changements. Le plus gros est de ne plus dépendre de Dropbox, qui limite le nombre d'utilisateurs à moins de payer un certain montant mensuel. Une alternative serait d'utiliser Google Drive, qui est très répandu parmi les utilisateurs Android.

\section{Conclusion}

Nous avons beaucoup apprécié faire ce projet, qui nous a permis d'apprendre énormément. La plus belle découverte est sans aucun doute Kotlin, qui est très agréable à utiliser et qui possède beaucoup de librairies et extensions plus qu'utiles sur Android.

\section*{Remerciements}

Nous tenons à remercier chaleureusement \textbf{Pauline Rossel} pour le logo, le \emph{splashscreen} et le \emph{parallax} dans la vue des détails, ils sont magnifiques !