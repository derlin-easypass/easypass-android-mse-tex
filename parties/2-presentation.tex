% !TEX ../report.tex

\section{Concepts fondamentaux de la suite \easypass{}}

\subsection{Présentation}
Le but d'\easypass{} est de proposer un password manager indépendant de tout serveur externe ou outil particulier. Les mots de passe sont stockés dans un fichier au format JSON. Pour garantir la sécurité de ce dernier, celui-ci est chiffré à l'aide d'un mot de passe selon l'algorithme AES-128-CBC tel qu'implémenté par l'outil opensource OpenSSL. Il est donc possible d'éditer ou récupérer ses mots de passe à l'aide d'une ligne de commande quelconque. \\
Les outils de la suite \easypass{} sont là pour offrir une couche \emph{user-friendly} à ce mécanisme de base. Ces outils se chargent de chiffrer/déchiffrer le fichier et d'en gérer le contenu (voir figure \ref{fig:schema-easypass}). 

Puisque nous ne faisons que manipuler un fichier, ce dernier peut être synchronisé ou sauvegardé avec de nombreux outils: Google Drive, Dropbox, etc. Sur Android, nous avons rendu l'utilisation de Dropbox obligatoire, mais ce n'est pas le cas sur les versions desktop.
Afin de rendre accessible le fichier depuis n'importe quel device, nous avons choisi de stocker le fichier sur Dropbox. 

\includeFigure{.8}{schema-easypass}{\easypass, architecture générale}

\subsection{Algorithme de chiffrement}

Le principe de \emph{password-based encryption} est de dériver une clé de chiffrement depuis un mot de passe fixe. La sécurité de cette méthode dépend alors de deux choses: a) la longueur du mot de passe, qui détermine également l'entropie de la clé générée, b) la PKDF (\emph{Password Key Derivation Function}), soit l'algorithme utilisé pour dériver une clé du mot de passe. Plus cette fonction prend du temps, plus une attaque brute-force deviendra difficile (manque de temps et de RAM).

\easypass{} utilise un chiffrement qui correspond aux lignes de commande OpenSSL suivante\footnote{AES est notamment utilisé par le gouvernement américain pour chiffrer ses données top secrètes. Ils utilisent cependant la variante 256-bits. Nous utilisons encore 128-bits pour rester compatible avec les autres outils de la suite (mise à jour prévue).}:

\begin{bashcode}
# encrypt
openssl enc -aes-128-cbc -salt -in file.json -out file.json.enc  

# decrypt
openssl enc -aes-128-cbc -d -a -in file.json.enc -out file.json 
\end{bashcode}


Pour mieux comprendre les mécanismes derrière le chiffrement AES-CBC ainsi que ses limites, nous recommandons notamment les ressources suivantes:

\begin{itemize}
    \item AES Encryption and Decryption with OpenSSL: \url{https://eclipsesource.com/blogs/2017/01/17/tutorial-aes-encryption-and-decryption-with-openssl/}
    \item Is password-based AES encryption secure at all? \url{https://crypto.stackexchange.com/questions/42538/is-password-based-aes-encryption-secure-at-all}
    \item Implémentation en Go du chiffrement \easypass{}: \url{https://github.com/derlin-easypass/easycmd-go/blob/master/crypto.go}
\end{itemize}

\section{Application Android}

fonctionnalités et captures aussi sur tablette !

Au premier lancement de l'appliation, l'utilisateur doit configurer l'accès vers son compte dropbox voir Fig. \ref{fig:login-dropbox}. L'application peut en suite utiliser cet accès pour créer et sauver les sessions de l'utilisateur. Lors de la première utilisation de l'application, aucune session n'est trouvée voir Fig. \ref{fig:login-dropbox}. Par défaut, l'application mobile easypass propose à l'utilisateur de créer la session TODO mettre le nom de la session. Elle sera chiffrée par le master password que l'utilisateur doit choisir à cet écran. La suite easypass permet à l'utilisateur de créer et de manipuler des sessions à travers d'autres outils. L'utilisateur peut choisir d'ouvrir un session  différente au même écran.

\begin{center}
	\begin{minipage}{.3\textwidth}
		\includeFigure{1}{loading}{} % Page de chargement.
	\end{minipage}
	\begin{minipage}{.3\textwidth}
		\includeFigure{1}{login-dropbox}{} % Choisir un compte Dropbox.
	\end{minipage}
	\begin{minipage}{.3\textwidth}
		\includeFigure{1}{choose-master-password}{} % Créer la session TODO mettre le nom de la session et le Master Password qui la chiffrera.
	\end{minipage}        
\end{center}

A présent, si l'application est relancée, le master password est demandé afin de déchiffrer la session TODO add fig. Le master password doit être compliqué afin de garantir la sécurité de l'application. Afin que l'utilisateur n'ait pas besoin de le rentrer à chaque utilisation, il est possible de le mettre en cache en cochant la case remember. L'utilisateur peut utiliser les méthodes d'authentifications fournies par Android (modèle, empreinte, etc). L'utilisateur peut choisir d'ouvrir une session différente. 

\begin{center}
	\begin{minipage}{.3\textwidth}
		\includeFigure{1}{loading}{} % Page de chargement.
	\end{minipage}
	\begin{minipage}{.3\textwidth}
		\includeFigure{1}{login}{} % Ouvre la session TODO mettre le nom de la session (ou une autre session selectionée par l'utilisateur) à l'aide du Master Password.
	\end{minipage}        
\end{center}

La vue principale ou l'activité principal affiche la liste de tous les comptes de l'utilisateur TODO add fig. Ils sont répertoriés par un Nom unique. L'utilisateur peut ajouter un nouveau compte grâce au bouton flottant.Différentes informations peuvent être spécifiées au moment de la création du compte ainsi que le mot de passe. L'application \easypass{} propose également de générer un password aléatoire selon différents critères. 

\begin{center}
	\begin{minipage}{.3\textwidth}
		\includeFigure{1}{TODO}{} % Exemple d'une liste contenant TODO comptes.
	\end{minipage}
	\begin{minipage}{.3\textwidth}
		\includeFigure{1}{new-account}{} % Création d'un nouveau compte.
	\end{minipage}
	\begin{minipage}{.3\textwidth}
		\includeFigure{1}{password-generator}{} % Générer un mot de passe.
	\end{minipage}        
\end{center}

La compte créé apparaît maintenant dans la liste des comptes TODO add fig. Lorsque l'utilisateur sélectionne un compte de la liste un pannel propose différentes actions.
Différents raccourcis permettent à l'utilisateur de copier les informations principales du compte sans l'ouvrir (Password, Username, Email). Un autre raccourcis permet d'afficher le mot de passe sauvé et de le copier. 

\begin{center}
	\begin{minipage}{.3\textwidth}
		\includeFigure{1}{liste}{} % Exemple d'une liste contenant TODO comptes.
	\end{minipage}
	\begin{minipage}{.3\textwidth}
		\includeFigure{1}{liste-edition}{} % Pannel lorsqu'un compte est sélectionné. Les options sont grisées lorsque les informations sont vides.
	\end{minipage}
	\begin{minipage}{.3\textwidth}
		\includeFigure{1}{password-shortcut}{} % Raccourcis permmettant de voir le mot de passe d'un compte.
	\end{minipage}        
\end{center}

Depuis ce pannel, l'utilisateur peut accéder à un nouvel écran et voir ou éditer les détails du compte sélectionné. Il est également possible d'afficher les détails du compte avec en cliquant longtemps sur un compte de la liste.

\begin{center}
	\begin{minipage}{.3\textwidth}
		\includeFigure{1}{details}{} % Les détails du compte.
	\end{minipage}
	\begin{minipage}{.3\textwidth}
		\includeFigure{1}{edition}{} % Ecran d'édition du compte.
	\end{minipage}        
\end{center}

Finalement l'utilisateur peut encore supprimer un compte depuis la liste principale. En faisant glisser le compte sur la gauche. Si un compte est supprimé ou modifié, il est possible de revenir en arrière en cliquant sur le bouton Undo.

\begin{center}
	\begin{minipage}{.3\textwidth}
		\includeFigure{1}{delete}{} % Exemple de suppression d'un compte.
	\end{minipage}
	\begin{minipage}{.3\textwidth}
		\includeFigure{1}{undo}{} % Affichage du bouton Undo pour annuler la dernière modification.
	\end{minipage}        
\end{center}

L'utilisateur peut modifier l'affichage de la liste principale pour trouver plus rapidement les comptes qu'ils cherchent. La liste principale peut être triée par nom ou date de création. Une fonction de recherche est également proposée pour retrouver un compte par son nom, son username ou son email. L'utilisateur peut choisir ses comptes favoris. Ils apparaîtront en haut de la liste. Les comptes favoris sont sauvés dans la session.

\begin{center}
	\begin{minipage}{.3\textwidth}
		\includeFigure{1}{tri}{} % Option de tri de la liste principale.
	\end{minipage}
	\begin{minipage}{.3\textwidth}
		\includeFigure{1}{search}{} % Option de recherche.
	\end{minipage}
\end{center}

La session ouverte dans l'application est toujours synchronisée avec Dropbox. Dans le cas où le téléphone n'a plus accès à internet, une copie de la session est conservée en locale avec la dernière synchronisation. L'application peut ainsi afficher les comptes mais sans les éditer.

\begin{center}
	\begin{minipage}{.3\textwidth}
		\includeFigure{1}{local-file}{} % Chargement de la dernière session locale sauvegardée.
	\end{minipage}
	\begin{minipage}{.3\textwidth}
		\includeFigure{1}{no-network}{} % Une icône en rouge est affichée lorsqu'il n'y a plus de réseau.
	\end{minipage}
	\begin{minipage}{.3\textwidth}
		\includeFigure{1}{edition-disabled}{} % Les actions d'éditions sont désactivées.
	\end{minipage}        
\end{center}

La suite \easypass{} permet de manipuler la session à travers plusieurs applications (Application Java, Ligne de commande, Application mobile). Dans le cas où la session serait modifiée par un autre programme. L'utilisateur peut forcer la synchronisation afin de récupérer les dernières modifications. Cette action est possible grâce au bouton sync.

\includeFigure{.5}{sync}{} % Force la synchronisation de la session locale avec la session sauvée sur Dropbox.

Les paramètres généraux de l'application peuvent être atteint depuis l'écran principale de l'application. Parmis les actions possibles, l'utilisateur peut :

Personnaliser la liste de caractères spéciaux utilisés lors de la génération du mot de passe. 
Modifier le master password de la session chargée.
Retirer le master password du cache.
Ouvrir l'application à nouveau avec une autre session.
Oublier l'accès au compte Dropbox de l'utilisateur.
Effacer le cache de l'application. Notamment la session local utilisé pour faire fonctionner l'application quand le téléphone n'a pas accès au réseau.

\begin{center}
	\begin{minipage}{.3\textwidth}
		\includeFigure{1}{settings}{} % Settings de l'application.
	\end{minipage}
	\begin{minipage}{.3\textwidth}
		\includeFigure{1}{custom-liste}{} % Edition de la liste des caractères spéciaux.
	\end{minipage}
	\begin{minipage}{.3\textwidth}
		\includeFigure{1}{new_password}{} % Edition du master password.
	\end{minipage}        
\end{center}


% exemple de référence
% blabla (see Figure \ref{fig:login}) blabla.

% exemple 1 figure
% \includeFigure{.5}{generate_password}{caption text}